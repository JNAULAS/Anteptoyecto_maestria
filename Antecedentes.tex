\subsection{DevOps}
DevOps nace por la necesidad de dar solución a las falencias existentes en el proceso de desarrollo de software debido al trabajo aislado entre las áreas involucradas como es los Desarrolladores e Infraestructura \cite{4599477}.

Con el afán de encontrar la solución a los problemas existentes se implementa los conceptos de metodologías agiles que permita avanzar con el cambio de mentalidad entre los grupos de trabajo lo cual permitiría encontrar ese dialogo necesario entre los equipos y que permita detectar y dar soluciones a casos que realizados por separado era muy difícil detectarlo a tiempo \cite{4599477}.

DevOps representa el cambio cultural de TI, centrándose en la entrega rápida de servicios de TI mediante la adopción de prácticas ágiles y eficientes en el contexto de un enfoque orientado a sistemas. DevOps enfatiza en la cultura de las personas y busca mejorar la colaboración entre los equipos de desarrollo y operaciones. Las implementaciones aplicadas en este enfoque también utilizan tecnología, especialmente herramientas de automatización que pueden aprovechar una infraestructura cada vez más configurable y dinámica desde una perspectiva del ciclo de vida del desarrollo de software \cite{AZAD2023107150}.